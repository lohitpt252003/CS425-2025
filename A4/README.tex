% README.tex
\documentclass[12pt]{article}
\usepackage[a4paper,margin=1in]{geometry}
\usepackage{hyperref}
\usepackage{parskip}

\title{CS425: Computer Networks\\Assignment 4: Routing Protocols (DVR and LSR)}
\author{Team Members: LOHIT P TALAVAR (210564), ANJALI MALOTH (210146)}
\date{\today}

\begin{document}
\maketitle

\section*{Objective}
Implement Distance Vector Routing (DVR) and Link State Routing (LSR) algorithms in C++ by reading an adjacency‐matrix representation of a network and producing per‐node routing tables.

\href{https://github.com/lohitpt252003/CS425-2025.git}{Assignment Repository on GitHub}

\section*{Repository Structure}
\begin{verbatim}
.
├── routing_sim.cpp     % C++ source implementing DVR and LSR
├── Makefile            % (optional) build rules
├── input1.txt          % sample adjacency matrix input
├── input2.txt          % sample adjacency matrix input
├── input3.txt          % sample adjacency matrix input
├── input4.txt          % sample adjacency matrix input
├── A4.pdf              % problem statement
└── README.tex          % This is LaTeX documentation
└── README.pdf          % This is documentation
\end{verbatim}

\section*{Prerequisites}
\begin{itemize}
  \item C++17‑compatible compiler (e.g. \texttt{g++})
  \item Standard C++ library (no external dependencies)
  \item Unix‑style shell (e.g. Bash) or equivalent
\end{itemize}

\newpage

\section*{Compilation}
\subsection*{Using Makefile}
\begin{verbatim}
make
\end{verbatim}

\subsection*{Manual}
\begin{verbatim}
g++ -std=c++17 routing_sim.cpp -o routing_sim 
\end{verbatim}
or
\begin{verbatim}
g++ routing_sim.cpp -o routing_sim
\end{verbatim}

\section*{Usage}
\begin{verbatim}
./routing_sim <input_file>
\end{verbatim}
\noindent\textbf{Example:}
\begin{verbatim}
./routing_sim inputfile.txt
\end{verbatim}

\section*{Input Format}
\begin{enumerate}
  \item First line: integer \(n\), number of nodes.
  \item Next \(n\) lines: each contains \(n\) space‑separated integers (the adjacency matrix):
    \begin{itemize}
      \item Off‑diagonal 0 indicates no link (treated as infinite cost = 9999).
      \item Value 9999 explicitly represents infinite cost.
      \item Diagonal entries must be 0.
    \end{itemize}
\end{enumerate}

\section*{Output Format}
\subsection*{Distance Vector Routing (DVR)}
\begin{verbatim}
--- Distance Vector Routing Simulation ---
--- DVR iteration k ---
Node i Routing Table:
Dest    Cost    Next Hop
...
--- DVR Final Tables ---
\end{verbatim}

\subsection*{Link State Routing (LSR)}
\begin{verbatim}
--- Link State Routing Simulation ---
Instructor: Adithya Vadapalli TA Incharge: Rishit and Yugul
CS425: Computer Networks A4: Routing Protocols (DVR and LSR)
Node i Routing Table:
Dest    Cost    Next Hop
...
\end{verbatim}

\section*{Algorithm Overview}
\subsection*{Distance Vector Routing}
\begin{enumerate}
  \item Initialize each node’s cost vector with direct link costs and next‑hop pointers.
  \item Repeatedly exchange vectors with neighbors and update via Bellman‑Ford until convergence:
  \[
    \text{if } dist_u[v] + dist_v[d] < dist_u[d] \text{ then }
      dist_u[d] = dist_u[v] + dist_v[d],\;
      nextHop_u[d] = nextHop_u[v].
  \]
  \item Terminate when no changes occur.
\end{enumerate}

\subsection*{Link State Routing}
\begin{enumerate}
  \item Each node knows the full topology (adjacency matrix).
  \item Run Dijkstra’s algorithm from each node to compute shortest paths.
  \item Backtrack predecessor array to determine first‑hop next‑hops.
\end{enumerate}

\section*{Code Workflow}
\subsection*{1. Reading the Graph}
\begin{itemize}
  \item \texttt{readGraph(const string\&)} opens the input file, reads \(n\), then the \(n\times n\) matrix.
  \item Off‑diagonal zeros (\(i \neq j\)) are converted to \(\texttt{INF}=9999\); diagonal stays 0.
\end{itemize}

\subsection*{2. Distance Vector Routing (\texttt{simulateDVR})}
\begin{itemize}
  \item Initialize \(\texttt{dist}[i][j]\) to direct costs, and \(\texttt{nextHop}[i][j]=j\) when a link exists.
  \item Iteratively “exchange” tables: for each node \(u\) and neighbor \(v\), relax
    \[
      dist_u[d] \;=\;\min(dist_u[d],\;dist_u[v]+dist_v[d]),
    \]
    updating \(\texttt{nextHop}_u[d]\) accordingly.
  \item Repeat until no update; print tables each iteration and final.
\end{itemize}

\subsection*{3. Link State Routing (\texttt{simulateLSR})}
\begin{itemize}
  \item For each source node \(s\):
    \begin{enumerate}
      \item Initialize \(\texttt{dist}[s]=0\), others = INF; \(\texttt{prev}[]=-1\).
      \item Run standard Dijkstra’s algorithm using the full adjacency matrix.
      \item After computing \(\texttt{dist}[]\) and \(\texttt{prev}[]\), backtrack from each destination \(d\) to \(s\) to find the first hop.
    \end{enumerate}
  \item Print routing table for each \(s\).
\end{itemize}

\subsection*{4. Main Function}
\begin{itemize}
  \item Parses command‑line argument for input file.
  \item Calls \texttt{readGraph}, then \texttt{simulateDVR} and \texttt{simulateLSR}.
\end{itemize}

\end{document}
